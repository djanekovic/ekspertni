\documentclass{article}
\usepackage[utf8]{inputenc}
\usepackage[T1]{fontenc}
\usepackage[croatian]{babel}
\usepackage{placeins}
\usepackage{mathptmx}
\usepackage{amsmath}
\usepackage{amssymb}
\usepackage{siunitx}
\usepackage{graphicx}
\usepackage[sorting=none]{biblatex}
%\addbibresource{literatura.bib}

\begin{document}

\begin{titlepage}
	\centering
	{\scshape\LARGE Sveučilište u Zagrebu\par}
	\vspace{0.5cm}
	{\scshape\Large Fakultet elektrotehnike i računarstva\par}
    \vfill
	{\Large\itshape Projekt kolegija Ekspertni sustavi \par}
	\vspace{0.5cm}
	{\huge\bfseries Rješavanje i analiza rješavanja igrice Minesweeper\par}
	\vspace{0.5cm}
	{\Large\itshape Darko Janeković, Jelena Nemčić \par}
	\vfill
	{\large Zagreb, 2020.}
\end{titlepage}

\section{Uvod}

Minesweeper je igra za jednog igrača koji se nalazi na polju koje sadrži skrivene mine. Igrač
u jednom potezu može ili označiti polje kao da se tamo nalazi mina ili ga otvoriti. Ukoliko je
polje označeno kao mina nije ga moguće otvoriti bez da se prethodno makne oznaka da se tamo
nalazi mina. Takva akcija je korisna ukoliko igrač sa sigurnošću zna da se na nekom polju
nalazi mina. S druge strane, igrač može otvoriti polje oko kojeg se ne nalazi niti jedna
mina, oko kojeg se nalazi $n$ mina te konačno polje na kojem se nalazi mina.

Otvaranje polja oko kojeg se ne nalazi niti jedna mina nastavit će rekurzivno otvarati susjedna
polja sve dok se ne otvori neko polje koje u susjedstvu ima minu. Otvaranje polja oko kojeg
se nalazi $n$ mina ispisat će na ploču brojku $n$. Igrač je izgubio ako je otvorio polje s minom, dok otvaranje zadnjeg polja koje nije mina
označava pobjedu.

U okviru ovog projekta bit će analizirane dvije taktike za rješavanje igrice Minesweeper.
U nastavku će biti analizirane dvije taktike rješavanja igrice Minesweeper pri čemu obje
taktike svode problem na  programiranje ograničenjima(engl. \textit{Constraint Programming}),
ali se postupak rješavanja razlikuje.

U prvom slučaju će se problem rješavati simplex metodom, dok će se u drugom slučaju isti
problem rješavati i tretirati kao problem zadovoljivosti(engl. \textit{Satisfaction Problem}).
Prednosti i nedostatci oba načina rješavanja bit će komentirani u nastavku.

\section{Modeliranje problema}

Moguće je pretpostaviti, bez gubitka općenitosti, da je ploča dimenzija $3 \times 3$ te da je
svaka vrijednost na ploči $m_{i, j} \in \{\bot, \top\}$. Ukoliko je vrijednost pojedinog polja
$\top$ tamo se ne nalazi mina, odnosno ako je vrijednost $\bot$ na tom polju se nalazi mina.
Skup $\mathcal{N}(m_{i, j})$ definira susjedstvo polja $m_{i, j}$, a s $n$ će biti označavan broj mina
u svim susjednim poljima polja $m_{i, j}$.

Nakon otvaranja polja $m_{i, j}$ agent u bazu znanja upisuje:
\begin{equation}
    \sum_{a \in \mathcal{N}(m_{i, j})} a = n
\end{equation}

\subsection{Zaključivanje}
Zaključivanje korištenjem ranije definiranih pravila bit će objašnjeno u nastavku. Stanje
svijeta prikazano je tablicom:

\begin{table}[ht]
    \centering
    \begin{tabular}{llll}
                           & 1                      & 2                      & 3                      \\ \cline{2-4}
    \multicolumn{1}{l|}{1} & \multicolumn{1}{l|}{}  & \multicolumn{1}{l|}{}  & \multicolumn{1}{l|}{}  \\ \cline{2-4}
    \multicolumn{1}{l|}{2} & \multicolumn{1}{l|}{1} & \multicolumn{1}{l|}{1} & \multicolumn{1}{l|}{1} \\ \cline{2-4}
    \multicolumn{1}{l|}{3} & \multicolumn{1}{l|}{0} & \multicolumn{1}{l|}{0} & \multicolumn{1}{l|}{0}  \\ \cline{2-4}
    \end{tabular}
\end{table}

\begin{align*}
    m_{1, 1} + m_{1, 2} = 1 \\
    m_{1, 1} + m_{1, 2} + m_{1, 3} = 1 \\
    m_{1, 2} + m_{1, 3} = 1 \\
\end{align*}

Iz čega se lako može zaključiti da vrijedi $m_{1, 1} = \bot$, $m_{1, 2} = \top$ i
$m_{1, 3} = \bot$. Drugim riječima jedina mina se nalazi na polju $m_{1, 2}$.

S druge strane, zaključivanja mogu biti nejednoznačna. Primjer stanja svijeta koji nije
jednoznačno rješiv nalazi se u tablici u nastavku:
\begin{table}[ht]
    \centering
    \begin{tabular}{llll}
                           & 1                      & 2                     & 3                     \\ \cline{2-4}
    \multicolumn{1}{l|}{1} & \multicolumn{1}{l|}{}  & \multicolumn{1}{l|}{} & \multicolumn{1}{l|}{} \\ \cline{2-4}
    \multicolumn{1}{l|}{2} & \multicolumn{1}{l|}{}  & \multicolumn{1}{l|}{} & \multicolumn{1}{l|}{} \\ \cline{2-4}
    \multicolumn{1}{l|}{3} & \multicolumn{1}{l|}{1} & \multicolumn{1}{l|}{} & \multicolumn{1}{l|}{} \\ \cline{2-4}
    \end{tabular}
\end{table}

\begin{equation*}
    m_{2, 1} + m_{2, 2} + m_{3, 2} = 1
\end{equation*}

S obzirom da agentu nije poznata nikakva druga informacija, agent na temelju ovog sustava ne
može ništa zaključiti i mora pogađati.

\subsubsection{Zaključivanje rješavanjem 3-SAT problema}
Prvi korak u zaključivanju rješavanje 3-SAT problema je svođenje problema u konjugiranu
normalnu formu(u nastavku CNF).

S obzirom da se oko polja $m_{2, 1}$ nalazi jedna mina prva klauzula će biti zadovoljena ako
je barem jedan sused mina, druga klauzula će biti zadovoljena ako barem jedno polje nije mina.
\begin{equation*}
    (m_{1, 1} \vee m_{1, 2}) \wedge (\neg m_{1, 1} \vee \neg m_{1, 2})
\end{equation*}

Slično razmišljanje kao i ranije moguće je primijeniti za ostala polja:
\begin{gather*}
    (m_{1, 1} \vee m_{1, 2} \vee m_{1, 3}) \wedge
        (\neg m_{1, 1} \vee \neg m_{1, 2}) \wedge
        (\neg m_{1, 2} \vee \neg m_{1, 3}) \wedge
        (\neg m_{1, 3} \vee \neg m_{1, 1}) \\
    (m_{1, 2} \vee m_{1, 3}) \wedge (\neg m_{1, 2} \vee \neg m_{1, 3})
\end{gather*}

Jednom kad je baza znanja predstavljena u CNF zaključivanje se izvodi teoremom dedukcije.
Primjerice, potrebno je zaključiti da se na polju $m_{2, 2}$ nalazi mina. Baza znanja $\Gamma$
proširuje se s $\neg m_{2, 2}$ te je potrebno pokazati da je $(\Gamma \wedge \neg m_{2, 2})$
nezadovoljiva.

Ukoliko je formula zadovoljiva, agent može biti siguran da se tamo nalazi mina te proširiti
bazu znanja s tom informacijom kao i označiti to polje kao opasno.


\subsubsection{Zaključivanje simpleks metodom}
Simpleks algoritam je metoda rješavanja problema u linearnom programiranju. Ovaj postupak se
može tumačiti kao pohlepna inačica 3-SAT postupka.

Ulaz u algoritam je niz nejednadžbi za koje je potrebno pronaći odgovarajuće nepoznanice
tako da je svaka nejednadžba zadovoljena.

%TODO: napiši kako smo napisali nejednadžbe

Ovaj algoritam za razlučivanje se ne ponaša kao i 3-SAT i to je za očekivati. Glavna razlika
između dva algoritma je da ukoliko 3-SAT algoritam nije siguran nalazi li se na nekom polju
mina pretpostavit će da se mina tamo nalazi. S druge strane, simpleks metoda će slučajni odabirom odabrati polja gdje će se nalaziti mine. Iz ovog razloga je 3-SAT puno bolji za rješavanje
ovog problema. Prednost simpleks algoritma nad 3-SAT algoritmom je u vremenskoj složenosti.

\section{Implementacija}
\section{Zaključak}
\end{document}
